%xhversion{v2.01 RdF} %PdJ,PdL,PdM,PdS,PdU,Pe6,PeI,PfB,PfD,RbN,RbP,RcL,RdC,RdD,RdF
%%%%%%%%%%%%%%%%%%%%%%%%%%%%%%%%%%%%%%%%%%%%%%%%%%%%%%%%%%%%%%%%%%%%%%%%%%%%%
%	Hinweise:
%%%%%%%%%%%%%%%%%%%%%%%%%%%%%%%%%%%%%%%%%%%%%%%%%%%%%%%%%%%%%%%%%%%%%%%%%%%%%
%	die auskommentierten 'usepackage'-Anweisungen sind
%	Alternativen oder Zusaetze, die iXH hier nicht,
%	aber vielleicht DU brauchen kannst - XH
%%%%%%%%%%%%%%%%%%%%%%%%%%%%%%%%%%%%%%%%%%%%%%%%%%%%%%%%%%%%%%%%%%%%%%%%%%%%%
%	XH Herstellungsprozess-Shellscript @15Apr17:
%	--------------------------------------------
%	#!/bin/sh
%	ifn="MeinLatexFile.tex"
%	latex $ifn
%	fn2="$(echo $ifn|sed s/.tex/.dvi/)"
%	fn3="$(echo $ifn|sed s/.tex/-pics.pdf/)"
%	  ### now dvipdf  $fn2 into $fn3 Container ...
%	dvipdf $fn2  $fn3
%	#dvipdf $(echo $ifn|sed s/.tex/.dvi/)
%	  ### now pdflateXing $ifn ... (zweimal - fuers Inhaltsverzeichnis)
%	pdflatex $ifn
%	pdflatex $ifn
%%%%%%%%%%%%%%%%%%%%%%%%%%%%%%%%%%%%%%%%%%%%%%%%%%%%%%%%%%%%%%%%%%%%%%%%%%%%%
\listfiles			%lists included files while processing 'pdflatex'
  %\documentclass[12pt,a4paper]{book}
  %\documentclass[11pt,a4paper]{article}
\documentclass[12pt,a4paper]{article}
  %\documentclass[12pt,a4paper]{report}

  %\usepackage{etex}		%gegen 'no more room for new dimen...' error xh@RaE1

	% encoding:
  %%\usepackage[latin1]{inputenc}
  %%\usepackage[ansinew]{inputenc}
  %%\usepackage[cp850]{inputenc}
  %\usepackage[utf8x]{inputenc}
\usepackage[utf8]{inputenc}
\usepackage[ngerman]{babel}
\usepackage[T1]{fontenc}  %\usepackage{amssymb}
\usepackage{amsmath}
%\usepackage{extarrows}	%\xleftrightarrow[obentext]{untentext}
\usepackage{wasysym}
\usepackage{pxfonts}
\usepackage{verbatim}
\usepackage{alltt}
\usepackage{moreverb}
\usepackage{graphicx}
\usepackage{wrapfig}
\usepackage{subfigure}
  %\usepackage{theorem}
  %\usepackage[dvips]{color}
  %\usepackage{lmodern}
  %\usepackage{textcomp}
\usepackage{multicol}		% 2-, 3-, ... -spaltige Formatierung mit 'multicols'
\usepackage{multirow}		% fuer 'tabular' - Tabellen
\usepackage{makeidx}
  %\usepackage{pdfpages}	% fuer 'includepdf' (iNimmMeistens 'includegraphics[page=1,...]')
\usepackage{mdwlist}		% f. 'compact lists' "itemize*", "enumerate*", "description*"
  %\usepackage{ulem}	... produziertma nFehler ban 'latex' run
\usepackage{longtable}		% fuer tabellen ueber mehrere Seiten
\usepackage{xcolor}
\definecolor{lightgrey}	{gray}{0.85}
\definecolor{llltgy}	{gray}{0.98}
\definecolor{lltgy}	{gray}{0.96}
\definecolor{ltgy}	{gray}{0.91}
\definecolor{grey}	{gray}{0.75}
\definecolor{dkgy}	{gray}{0.35}
\definecolor{ddkgy}	{gray}{0.17}
\definecolor{dddkgy}	{gray}{0.07}
\definecolor{blk}	{gray}{0.99}
\definecolor{lltgn}	{rgb}{0.96,1.0,0.96}
\definecolor{ltgn}	{rgb}{0.91,1.0,0.91}
\definecolor{mdgn}	{rgb}{0.7,1.0,0.7}
\definecolor{dkgn}	{rgb}{0.0,0.7,0.0}
\definecolor{ddkgn}	{rgb}{0.0,0.45,0.0}
\definecolor{dddkgn}	{rgb}{0.0,0.25,0.0}
\definecolor{mdye}	{rgb}{0.95,0.95,0.60}
\definecolor{ltye}	{rgb}{0.98,0.98,0.90}
\definecolor{lltor}	{rgb}{0.97,0.94,.87}
\definecolor{ltor}	{rgb}{0.95,0.85,.66}
\definecolor{red}	{rgb}{1.0,0.0,0.0}
\definecolor{dkred}	{rgb}{0.8,0.0,0.0}
\definecolor{ltrd}	{rgb}{1.0,0.8,0.8}
\definecolor{ltbu}	{rgb}{0.8,0.8,1.0}
\definecolor{mdbu}	{rgb}{0.7,0.7,1.0}
\definecolor{bu}	{rgb}{0.0,0.0,1.0}
\definecolor{dkbu}	{rgb}{0.0,0.0,0.6}
\definecolor{ddkbu}	{rgb}{0.0,0.0,0.45}
\definecolor{ddkrd}	{rgb}{0.25,0.0,0.0}
\definecolor{dkrd}	{rgb}{0.70,0.0,0.0}
\definecolor{indigo}	{rgb}{0.2,0.1,0.9}

\usepackage{listings}
\lstset{language=C}
\lstset{basicstyle=\tiny}
  %\lstset{basicstyle=\small}
  %\lstset{basicstyle=\normalsize}
\lstset{backgroundcolor=\color{lightgrey}}
\lstset{showstringspaces=false}
\lstset{breaklines=true}
  %\lstset{tabsize=4}
\lstset{morecomment=[l][\color{dkgn}]{\%},%
	morecomment=[s][\color{dkgn}]{/*}{*/}}
\lstset{numbers=left}

\usepackage{fancyhdr}
  %\usepackage{framed}		%'\begin{framed}' ... '\end{framed}', schautAusWiePartezettel:-)
\usepackage{hyphenat}		%fuer '\hyph{}'
  %\usepackage{lastpage}	%fuer '\pageref{LastPage}' - **funzt nid bei allen**
\usepackage{url}		%fuer '\url{...}'

% lscape oder pdflscape: ('landscape' == Querformat)
\usepackage{lscape}
  %\usepackage{pdflscape}
\usepackage{rotating}		%f. 'rotate' und 'turn'
\usepackage[active]{pst-pdf}
\usepackage{pst-circ}
\usepackage{pst-plot}
\usepackage{pst-uml}
  %\usepackage{calc}
\usepackage{fp}
  %\usepackage[official]{eurosym}
\usepackage[gen]{eurosym}

	% YHs Raender links 30mm rechts 25mm einstellen:
\setlength{\hoffset}	{30mm-1in}
\setlength{\oddsidemargin}{0pt}		%bei doppelseitigem Druck umstellen!
\setlength{\textwidth}	{\paperwidth-55mm}

\setlength{\topmargin}	{0pt}
\addtolength{\voffset}  {-16.2mm}
\addtolength{\textheight}{45mm}

\setcounter{tocdepth}{4}		%bringt auch 'paragraph{titel}' ins Inhaltsverzeichnis

\newcommand{\cmnt}[1]{}			%eigene Kommentier-Funktion \cmnt{ ...Kommentar... }
\newcommand\tbs{\textbackslash}		%'\textbackslash{}' isma z'long zan tippen ;-)
\newcommand\dtbs{\textbackslash\textbackslash}	% -dito-
%
\definecolor{ydkbu}{rgb}{0.0,0.0,0.6}	% YHs blaue Schriftfarb
\newcommand{\yhbu}[0]{\color{ydkbu}}	% Macro fuer schreibfaulen XH
\definecolor{corrclr}{rgb}{0.7,0.2,0.2}		% XHs Korrekturen-Farb ...
\newcommand{\korr}[0]{\color{corrclr}\fontsize{8pt}{9pt}\selectfont\bf} %plus Faulheitsmacro
\makeindex

	%/* Line Spacing: */
\usepackage{setspace}
% \newcommand{\mylinespacing}[0]{\singlespace}
\newcommand{\mylinespacing}[0]{\onehalfspace}	% 1,5-ZeilenAbstand
% \newcommand{\mylinespacing}[0]{\doublespace}

% /*Font Family:*/
%\renewcommand*{\familydefault}{\rmdefault}	%klassisches 'Roman' (statt MicroMurx...)
\renewcommand*{\familydefault}{\sfdefault}	%klassisches 'Helvetica' statt 'MS-Arial'

%===================================================
\begin{document}
%\addtocontents{toc}{\protect\begin{multicols}{2}} %-fuer mehrspaltiges Inh.Verz

\newcommand\logoB[1]{%
	%dieses Macro '' zeichnet das "neue" HTL Logo mithilfe der
	% 'ps-tricks' Pakete/Anweisungen; Parameter#1 bestimmt die "Dicke"
	% der Balken; die "Groesse" bitte mit '\scalebox{factor}{logoB{0.12}}',
	% die Grundlinie mit '\raisebox{pos}{logoB{0.12}}' einstellen;
	% die Farbgebung spezifiziert man HIER:
  \definecolor{lobu}{rgb}{0.05,0.05,0.50}
  \definecolor{hibu}{rgb}{0.20,0.20,0.70}
  \definecolor{loye}{rgb}{0.85,0.75,0.36}
  \definecolor{hiye}{rgb}{0.99,0.92,0.00}
  \definecolor{logn}{rgb}{0.00,0.65,0.20}
  \definecolor{hign}{rgb}{0.00,0.79,0.30}
  \definecolor{lord}{rgb}{0.66,0.00,0.00}
  \definecolor{hird}{rgb}{0.89,0.00,0.00}%
  \resizebox{11.5mm}{!}{%
  \begin{pspicture}[showgrid=false](-1,-1)(1,1)
	\SpecialCoor	%das erlaubt PS -Berechnungen mit dem '!'; hier zur "DickenSkalierung"
	\pspolygon[linewidth=0.1pt,linestyle=none,fillcolor=lobu,fillstyle=solid]%
		(-#1, -1.00)( #1, -1.00)( 1.00, -#1)(! 1.00 #1 2 mul sub -#1)
	\pspolygon[linewidth=0.1pt,linestyle=none,fillcolor=hibu,fillstyle=solid]%
		(! 1.00 #1 2 mul sub          -#1)(! 1.00 #1 3 mul sub   0.00)%
		(! -#1         -1.00 #1 2 mul add)(-#1,-1.00)

	\pspolygon[linewidth=0.1pt,linestyle=none,fillcolor=hiye,fillstyle=solid]%
		( 1.00, -#1)( 1.00, #1)( #1, 1.00)(! #1   1.00 #1 2 mul sub)
	\pspolygon[linewidth=0.1pt,linestyle=none,fillcolor=loye,fillstyle=solid]%
		(! #1    1.00 #1 2 mul sub)(! 0.00   1.00 #1 3 mul sub)%
		(! 1.00 #1 2 mul sub   -#1)( 1.00, -#1)

	\pspolygon[linewidth=0.0pt,linestyle=none,fillcolor=hign,fillstyle=solid]%
		( #1, 1.00)( -#1, 1.00)(-1.00, #1)(! -1.00 #1 2 mul add   #1)
	\pspolygon[linewidth=0.0pt,linestyle=none,fillcolor=logn,fillstyle=solid]%
		(! -1.00 #1 2 mul add   #1)(! -1.00 #1 3 mul add    0.00)%
		(! #1    1.00 #1 2 mul sub)( #1, 1.00)

	\pspolygon[linewidth=0.1pt,linestyle=none,fillcolor=lord,fillstyle=solid]%
		(-1.00, #1)(-1.00, -#1)(-#1, -1.00)(! -#1    -1.00 #1 2 mul add)
	\pspolygon[linewidth=0.1pt,linestyle=none,fillcolor=hird,fillstyle=solid]%
		(! -#1   -1.00 #1 2 mul add)(! 0.00   -1.00 #1 3 mul add)%
		(! -1.00 #1 2 mul add    #1)(-1.00, #1)
	\NormalCoor
  \end{pspicture}%
  }%
}

\newcommand{\HtlHeader}[0]{%
	\hspace*{-11mm}%
	\raisebox{-1mm}{\logoB{0.12}}%
	%\includegraphics[width=10.3mm]{pics/logo}
	\hspace*{2mm}%
	\parbox[b]{110mm}{\flushleft
		{\fontsize{20pt}{20pt}\selectfont\bf HTL}
		{\fontsize{16.2pt}{16.2pt}\selectfont\color{teal}\bf anichstrasse}
		\\[-4.05mm]{\color{darkgray}\rule{110mm}{0.5pt}}
		\\[-2.24mm]{\fontsize{7pt}{7pt}\selectfont\color{darkgray}
			Elektronik $\cdot$ Elektrotechnik $\cdot$
			Maschinenbau $\cdot$ Wirtschaftsingenieure
			\rule{0pt}{0mm}
		%\vspace*{1.1mm}
		}
	}%
	\hspace*{5mm}%
	%\raisebox{-0.2mm}{ \includegraphics[width=25mm]{pics/HTLgenlogo02}}
	\\[-1.5mm]\rule{\textwidth}{0.5pt}
	%\hfill
}%HtlHeader



	%/*Header-Einstellung*/
\pagestyle{fancy}
\fancyhf{}
\renewcommand{\sectionmark}[1]{\markright{#1}}
\renewcommand{\subsectionmark}[1]{\markright{#1}}
\renewcommand{\subsubsectionmark}[1]{\markright{#1}}
\lhead{}
\chead{\HtlHeader{}}
\rhead{}
\lfoot{Alexander Beiser / Marcel Huber}
\cfoot{\thesection-\rightmark}
%\cfoot{\thesubsubsection-\rightmark}
\rfoot[\thepage]{\thepage/\pageref{LastPage}}
\setlength{\headwidth}	{1.0\textwidth}
\setlength{\headheight}{6mm}
\renewcommand{\headrulewidth}{0.0pt}
\renewcommand{\footrulewidth}{0.33pt}




	%/* Deckblatt */
\begin{titlepage}
 \begin{center}
   \begin{minipage}{\linewidth}
   \begin{center}
	\vspace*{-14mm}
	{\fontsize{25pt}{25pt}\selectfont\bf DIPLOMARBEIT}
	\\[19mm]{\fontsize{20pt}{20pt}\selectfont\color{blue}\textbf{\textsc{JavaChess, ChessPI AndChess}}}
	\\[15mm]{\fontsize{12.4pt}{12.4pt}\selectfont\bf
		Höhere Technische Bundeslehr- und Versuchsanstalt Anichstrasse}
	\\[ 5mm]\rule{132mm}{1.0pt}
	\\[ 4mm]{\fontsize{12.4pt}{12.4pt}\selectfont\bf Abteilung}
	\\[ 5mm]{\fontsize{12.4pt}{12.4pt}\selectfont\bf Elektronik \& Technische Informatik}
	\\[24mm]{\hspace*{2mm}\parbox{154mm}{\fontsize{12.4pt}{12.4pt}\selectfont
	  \parbox[t]{75mm}{
		Ausgefuehrt im Schuljahr 2017/18 von:
		\\[5.0mm]Alexander Beiser 5CHEL
		\\[2.5mm]Marcel Huber 5CHEL 
	  }
	  \hspace*{6mm}
	  \parbox[t]{50mm}{
		Betreuer/Betreuerin:
		\\[5.0mm]Ing. MSc. Signitzer Markus
	  }
	  \\[14mm]{Innsbruck, am 30.10.2017}
	  \\[16mm]\rule{150mm}{0.5pt}
	  \\[ 8mm]
	  \parbox[t]{75mm}{
		Abgabevermerk:
		\\[3.25mm]Datum:
	  }
	  \hspace*{6mm}
	  \parbox[t]{50mm}{
		Betreuer/in:
	  }
	}}
   \end{center}\hfill
   \end{minipage}
 \end{center}
\end{titlepage}


\addtocounter{page}{1}


%====================================================================================
%Liebe LaTeXniker!
%hierher kaeme das Inhaltsverzeichnis, empfehlenswerterweise mit Seitenwechsel
%\clearpage	%erzwingt Ausdruck noch ungedruckter 'floats'
%\vfill		%fuellt die Seite mit Leerraum auf
%\newpage	%erzwingt Seitenumbruch
%\tableofcontents
%====================================================================================


\vfill
\newpage
\tableofcontents










%====================================================================================
%Best comment of all time!!!
\cmnt{
	Hier anfangs the Document Text.
	("\cmnt" isa self written very simple Macro for Kommentare:
	\newcommand{\cmnt}[1]{ }
                      !    !  !
                      !    !  +--- what to do, here also nix
                      !    +------ number of Parameters: se Kommentar-Text
                      +----------- name of new command
	)
	
	The "\yhbu" colored Sections are Vorgaben (recommendations) by AV YH.
	All se blue Zuig have to verschwind in se final version of your Diplomschrift(DS).
	(iXH recommend not to translate the words 'Diplomschrift' or 'Diplomarbeit'
	into 'diploma document', 'diploma project' or such Kas,
	because the original words are defined in se Austrian Law, Verordnungen
	and derlei rechtlix Plunder; so it is like an Eingenname,
	which we also dont uebersetz:
	You dont traslate 'HTL' to 'UTEC' (upper technical education corporation)
	or
	'Hansi' ('H', 'ans', 'i') 'Meier' ('M' and 'eier') into 'Ageoneeye Emeggs'
	oder?? )

	("\yhbu" Macro see above; is also a selber-defined macro:
	\definecolor{ydkbu}	{rgb}{0.0,0.0,0.5}   %make a Farb-Name
	               !          !     !   !   +--- Blau-Anteil
	               !          !     !   +------- Gruen-Anteil
	               !          !     +----------- Rot-Anteil
	               !          +----------------- Farbmodell 'RGB'
	               +---------------------------- name of se new Farb
	\newcommand{\yhbu}[0]{\color{ydkbu}}  %define se new macro
	              !    !       +--------- schreib des in LaTeX-Text eini
	              !    +----------------- 0 = null parameters, also keine
	              +---------------------- name of macro
	-> "\yhbu" gets replaced by "\color{ydkbu}"
	(what does this bring?:
	you can later change se color for all se "yhbu" parts gemeinsam
	without wurschtling through the whole document;
	also wenn mir die Farb no nid gfallt, aendris uanfoch in Macro)

	You can change the
		line spacing (einzeilig, 1.5zeilig und so)
	by schreibing one of
	   \newcommand{\mylinespacing}[0]{\singlespace}
	   \newcommand{\mylinespacing}[0]{\onehalfspace}
	   \newcommand{\mylinespacing}[0]{\doublespace}
	and using '\mylinespacing' in se document preamble (=header) part
	and the
		font family
		(Roman(serif) or se ugly Arial/Helvetica(sanserif))
	writing
	   \renewcommand*{\familydefault}{\rmdefault}
	   \renewcommand*{\familydefault}{\sfdefault}
	weiter oben in se preamble (isch bei se Macos oben)

	Be free to change se Deckblatt
	and se HTL-Header (dann isches aber nimma 'der HTL-Header'! ...and YH will fauch on you)

	mirXH persoenlich gfollaz besser min HTL-Header auf jedn Blattl
	(command: \lhead{\HtlHeader})
	he ghearat holt black-and-white, because colors gelten als 'kitschig' (kitchy)
	aber me asks jo nobody (i am only a small Wuerschtl from se behindmountain (Hintelgebilge))

	ATTENTION!
	This 'Inhaltsverzeichnis'
	does NOT pass zu se document text here.
	i just gewaltsam made it look like YH's Vorlage.
	iXH also dont understand,
	why it is mitten im Dokument anstatt at se beginnig or at se end,
	why se headline (Ueberschrift) is not at it's first page oben,
	but ganz lonely on the page before,
	and why it starts with Kapitel 2.1 anstatt 1.0
	and on page 8 statt 1 oder 2;
	it suggeriers that Loesungswege, Nutzwertanalysen, Grobwentwurf, Feinentwurf
	und Implementierung auf derselben Seite 11 Platz haben,
	Fertigungsdokumentation plus Gebrauchsanweisung (S.14)
	sowie das Pflichtenheft(S.18) nur 1 Seite lang sein brauchen,
	se p.22 must be empty
	and se Projektterminplanung erst am Ende des Projektes nach der
	Feststellung der Projekterfahrungen erfolgt.
	but i am eben a dumms kind, a small Wuerschtl...


}%cmnt
%====================================================================================

\mylinespacing
{




%====================================================================================
\clearpage\vfill\newpage%\part{U-Lektionen \dq{}embedded Systems\dq{}}
%====================================================================================

\addcontentsline{toc}{section}{Erklaerung der Eigenstaendigkeit der Arbeit}
\section*{\Large\sc Erklaerung der Eigenstaendigkeit der Arbeit}
	\hfill\\[ 8mm]
	EIDESSTATTLICHE ERKLÄRUNG
	\\[3mm]
\begin{spacing}{1.5}
	\noindent%
	Ich erklaere an Eides statt, dass ich die vorliegende Diplomarbeit selbstaendig und
	ohne fremde Hilfe verfasst, andere als die angegebenen Quellen und Hilfsmittel
	nicht benutzt und die den benutzten Quellen woertlich und inhaltlich entnommenen
	Stellen als solche erkenntlich gemacht habe.
\end{spacing}\hfill
	\\[12mm]
	\parbox[b]{52mm}{
		\rule{50mm}{0.2pt}\rule{0pt}{25mm}
		\\\hspace*{6mm}{Ort, Datum}
		\\[0mm]
	}
	\hfill
	\parbox[b]{72mm}{
		\rule{70mm}{0.2pt}\rule{0pt}{25mm}
		\\\hspace*{6mm}{Verfasser, Verfasserinnen}
		\\\hspace*{6mm}{Vor- und Zunamen}
	}







%======================================================================================
%\clearpage\vfill\newpage
%======================================================================================
\section{Zusammenfassung des Projektergebnisses}
 \subsection{Kurzfassung /Abstract}
 
	Alexander Beiser und Marcel Huber entwickelten im Zuge ihrer Diplomarbeit 2017/18 ein Schachspiel, welches auf einem PC, Android-Smartphone und Raspberry PI spielbar ist. Das Schachspiel basierte auf einem bereits von ihnen geschriebenen rohen ,,Gerüst''. \\
	Dieses Spiel wurde mit der Programmiersprache Java entwickelt, weiteres war für den Raspberry PI ein Gehäuse zu Designen und mit einem 3D Drucker zu realisieren. Um das Spielvergnügen für den Raspberry PI auch Unterwegs zu ermöglichen wurde eine Akkusteuerung entworfen und realisiert. \\
	\subsubsection{Alexander Beiser}
	Alexander Beiser war für große Teile des Backends zuständig. Ein Hauptteil bestand aus der Entwicklung einer Chess Engine, also eines Zugmechanismusses welcher speziell für die ebenso von Alexander Beiser erschoffene Künstliche Intelligenz performiert wurde. Er entwickelte auch die Akkusteuerung für den Raspberry PI und designte das Gehäuse.
	
	\subsubsection{Marcel Huber}
	Marcel Huber war für weite Teile des Frontend Bereichs zuständig, die anderen Schwerpunkte bestanden in der Entwicklung des Netzwerkalgorithmus und der Entwicklung des Schachspiels für Android Geräte. 
 
	\vfill
	\newpage	
	
 \subsection{Projektergebnis}
	{\yhbu
	Allgemeine Beschreibung, was vom Projektziel umgesetzt wurde, in einigen kurzen Sätzen.
	Optional Hinweise auf Erweiterungen.
	Gut machen sich in diesem Kapitel auch Bilder vom Gerät (HW) bzw. Screenshots (SW).
	\\[1mm]
	Liste aller im Pflichtenheft aufgeführten Anforderungen,
	die nur teilweise oder gar nicht umgesetzt wurden (mit Begründungen).
	}










%======================================================================================
\clearpage\vfill\newpage
%======================================================================================
	\vspace*{-10mm}\noindent%
	{\Large\bf Inhaltsverzeichnis}\\
 \renewcommand{\theenumii}{\arabic{enumii}}
 \renewcommand{\labelenumii}{\theenumi.\theenumii}
 \renewcommand{\theenumiii}{\arabic{enumiii}}
 \renewcommand{\labelenumiii}{\theenumi.\theenumii.\theenumiii}
\cmnt{
{\yhbu
	Formale und sprachliche Aspekte \dotfill i		\\
	Zitierregeln \dotfill iv				\\
	Erklaerung der Eigenstaendigkeit der Arbeit \dotfill vii	\\
	Zusammenfassung des Projektergebnisses \dotfill viii		\\
	\phantom{11}Kurzfassung /Abstract \dotfill viii				\\
	\phantom{11}Projektergebnis \dotfill viii					\\
	1 Einleitung \dotfill 1					\\
	2 Vertiefende Aufgabenstellung \dotfill 1		\\
	\phantom{11}2.1 Schülername 1 \dotfill 1				\\
	\phantom{11}2.2 Schülerinnenname 2 \dotfill 1			\\
	3 Systemdokumentation \dotfill 2			\\
	\phantom{11}3.1 Lösungsweg \dotfill 2				\\
	\phantom{111}3.1.1 Gewählte Lösung \dotfill 2			\\
	\phantom{111}3.1.2 Alternative Lösungen (sollten Alternativen besprochen worden sein) \dotfill 2	\\
	\phantom{11}3.2 Grobentwurf \dotfill 2				\\
	\phantom{11}3.3 Feinentwurf \dotfill 2				\\
	\phantom{11}3.4 Implementierung \dotfill 2				\\
	\phantom{111}3.4.1 Sourcecode \dotfill 2				\\
	\phantom{111}3.4.2 Gesamtschaltplan und Fertigungsunterlagen \dotfill 3	\\
	\phantom{111}3.4.3 Verwendete Technologien und Entwicklungswerkzeuge \dotfill 3	\\
	\phantom{111}3.4.4 Testfälle \dotfill 4				\\
	\phantom{111}3.4.5 Test- und Messergebnisse \dotfill 4		\\
	4 Fertigungsdokumentation \dotfill 5			\\
	5 Benutzerdokumentation \dotfill 5			\\
	\phantom{11}5.1 Installationsanleitung \dotfill 5			\\
	\phantom{11}5.2 Anwendungsbeispiele \dotfill 5			\\
	\phantom{11}5.3 Referenzhandbuch \dotfill 5				\\
	\phantom{11}5.4 Fehlermeldungen und Hinweise auf Fehlerursachen \dotfill 5		\\
\clearpage\vfill\newpage{}\noindent%
	I. Abbildungsverzeichnis \dotfill I		\\
	II. Tabellenverzeichnis \dotfill I		\\
	III. Literaturverzeichnis \dotfill II		\\
	Anhang \dotfill IV				\\
	6 Pflichtenheft \dotfill IV			\\
	\phantom{11}6.1 Funktionale Anforderungen \dotfill IV	\\
	\phantom{11}6.2 Schnittstellen \dotfill IV			\\
	\phantom{11}6.3 Abnahmekriterien \dotfill IV		\\
	\phantom{11}6.4 Dokumentationsanforderungen \dotfill IV	\\
	\phantom{11}6.5 Qualitätsstandards \dotfill V		\\
	\phantom{11}6.6 Abwicklungsprozess \dotfill V		\\
	7 Zusammenfassung \dotfill VI			\\
	\phantom{11}7.1 Schlussfolgerung / Projekterfahrung \dotfill VI	\\
	\phantom{11}7.2 Projektterminplanung \dotfill VI		\\
	\phantom{11}7.3 Projektpersonalplanung und Kostenplanung \dotfill VII	\\
	\phantom{111}7.3.1 Projektkostenplan \dotfill VII		\\
	\phantom{111}7.3.2 Arbeitsnachweis Diplomarbeit \dotfill VII	\\
	\phantom{111}7.3.3 Leistungscontrolling \dotfill VII		\\
}
}	%



%=================================================================================
\clearpage\vfill\newpage{}
%=================================================================================

\section{Lizenz:}

Das Schachprogramm wird unter der ,,Creative Commons Attribution-NonCommercial-ShareAlike 4.0 International Public License'' entwickelt. \\
Dies räumt jeden Menschen folgende Rechte ein: 
\begin{itemize}
	\item{\textbf{Teilen:} Das Programm darf frei kopiert und weiterverteilt werden.}
	\item{\textbf{Verändern:} Das Programm darf frei verändert werden. Somit dürfen natürlich Verbesserungen implementiert werden.}
\end{itemize}

Allerdings muss dies unter den folgenden Bedingungen geschehen:
\begin{itemize}
	\item{\textbf{Zuschreibung:} Man muss die Namen der Entwickler entsprechend anführen und angeben, ob Veränderungen gemacht wurden.}
	\item{\textbf{Nicht kommerziell:} Das Programm darf nicht kommerziell benützt werden.}
	\item{\textbf{Gleiche Lizenz:} Sobald Veränderungen gemacht wurden, muss die original Lizenz weiter verwendet werden. Auch darf nicht von der obigen genannten Lizenz abgewichen werden.}
	\item{\textbf{Gesetzeskonform:} Das Programm darf nicht so verändert werden, dass die Nutzung illegal wird.}
\end{itemize}





%=================================================================================
\clearpage\vfill\newpage{}
%=================================================================================
\setcounter{section}{0}
\section{\sc Einleitung}

	Alexander und Marcel sind beide begeisterte Schachspieler, womit die Entwicklung eines Schachspiels nahe liegt. Gegen Ende des 4.Jahres der HTL trafen sie die Entscheidung ein Schachspiel selber zu entwickeln und keine Diplomarbeit von einer Firma anzunehmen. Für die Entwicklung ihres Schachspiels, werden sie von ,,Elektrotechnik Beiser'' unterstüzt. Diese Firma übernimmt etwaige anfallende kosten für die Hardwarekomponenten.\\
	Anfang der 5.Klasse der HTL kamen die Sondierungsgespräche mit ihrem Betreuer Ing. MSc. Signitzer Markus, 
welcher Ihnen sagte was im Zuge dieser Diplomarbeit alles erledigt werden müsse. \\
Durch die Gespräche kam man zum Schluss, dass für die Diplomarbeit ein Schachspiel in Java entwickelt werden muss und dieses auf einen RaspberryPI, als auch auf Android Geräte potiert werden soll. Weiteres wird eine Akkusteuerung für den RaspberryPI entwickelt und ein Gehäuse designed und mittels Schuleigenen 3D-Drucker ausgedruckt. \\
	Die GUI des Spiels soll mit JavaFX erstellt werden. Das Spiel soll gegen eine selbstentwickelte Künstliche Intelligenz spielbar sein, im Hot Seat Modus oder im Local Area Network. \\
	Im Hot Seat Modus spielt man auf einem PC abwechselnt die Partien.
	Details werden in einem Pflichtenheft festgehalten, dieses Pflichtenhet befindet sich im Anhang.
	

	{\yhbu
	In der Einleitung wird erklärt,
	wieso man sich für dieses Thema entschieden hat.
	(Zielsetzung und Aufgabenstellung des Gesamtprojekts,
	fachliches und wirtschaftliches Umfeld)
	}
\section{\sc Vertiefende Aufgabenstellung}
 \subsection{Alexander Beiser}
 	Überarbeitung des Schachmattalgorithmus, Entwicklung der Zugmechanik und Entwicklung einer künstlichen Intelligenz. \\
Implementierung des Schachspiels auf den Raspberry-PI, gleichzeitige designen des Gehäuses für den Raspberry-PI und Entwicklung der Akkuansteuerungsschaltung. 
	
 \subsection{Marcel Huber}
	Entwicklung der Netzwerkfähigkeit und Implementierung von Java FX.
Verbesserung und Weiterentwicklung der audio- und visuellen Gestaltung.
Entwicklung der Android-App und Fertigung des Raspberry PI-Gehäuses


%=================================================================================
\clearpage\vfill\newpage{}
%=================================================================================

\section{Schach, eine Erklärung}
\subsection{ist Schach?}
Um den Aufbau des Programmes nachzuvollziehen zu können, sollten die Grundregeln des Schachspiels geläufig sein. Hier haben wir versucht, die wichtigsten kurz Zusammenzufassen. \\
Was ist Schach? \\
Schach ist ein strategisches Brettspiel, indem es darum geht die feindliche Seite zuschlagen. Die feindliche Seite hat asbald verloren, wann der König gefallen ist. \\
Der Name Schach kommt aus dem persichen ,,Schah'' und bedeutet so viel wie König, woher der Name ,,königliches Spiel'' stammt. \\
Ursprünglich wurde das Spiel vermutlich in Nordindien erfunden und kam im Zuge der islamischen Expansion, von 630 bis ca. 750, nach Europa.


\subsection{Spielregeln:}

Nach der 1.Erklärung was Schach ist, kommen wir zu den Spielregeln.
Schach wird aufm einem 8*8 karierten Feld gespielt. Die Nummerrierung erfolgt horizontal durch das Alphabet, a bis h und vertikal durch Ziffern, 1 bis 8.
Zu Beginn gibt es zwei Teams, meist Weiß und Schwarz, mit jeweils 16 Figuren.
Folgende Figuren sind zu Beginn am Feld:
\begin{itemize}
	\item{8 Bauern}
	\item{2 Springern}
	\item{2 Läufern}
	\item{2 Türmen}
	\item{1 Dame}
	\item{1 König}
\end{itemize}

Das Ende des Spiels erfolgt entweder durch Schachmatt, Aufgabe oder durch ein Remis/Patt. Schachmatt bedeutet, dass der König bedroht wird und es dem Spieler nicht mehr möglich ist den König aus dieser Position zu befreihen.
Patt Möglichkeiten:
\begin{itemize}
	\item{ entsteht wenn eine der Parteien keinen legalen Zug mehr ausführen kann }
	\item{Durch ein ,,technisches Remis'', wenn außer den beiden Königen nur mehr ein Läufer oder Springer am Feld ist.}
	\item{Wenn 50 Züge lang keine Spielfigur geschlagen oder ein Bauer bewegt wurde und der am Zug befindliche Spieler das Remis verkündet.}
	\item{Wenn eine identische Stellung drei mal mit identischen Zugmöglichkeiten mindestens drei mal vorkommt, kann ein Spieler ein Remis beantragen.}
	\item{Wenn 75 Züge lang keine Figur geschlagen bzw. ein Bauer gezogen wurde oder fünmal diesselbe Stellung mit gleichen Zugmöglichkeiten entstanden ist. Hierbei wird das Remis automatisch vom Schiedsrichter/Schachspiel eingeleitet.}
\end{itemize}

Nun folgen die Zugregeln:

\subsubsection{Zugregel Bauer:}
\begin{itemize}
	\item{Bauer darf einen Schritt nach vorne ziehen, wenn das Feld leer ist}
	\item{Befindet sich der Bauer in der Ausgangsposition und wurde noch nicht gezogen, kann er auch wahlweise zwei Schritte vorrücken.}
	\item{Der Bauer schlägt vorwärts diagonal ein Feld.}
	\item{Spezialzug: ,,En Passant''. Dies kann er als einzige Spielfigur, wenn ein feindlicher Bauer zuvor einen Doppelschritt gemacht hat und somit den eigenen Bauern die Option nimmt, den gegnerischen Bauern anzugreifen. Falls er ausgeführt wird, ist der feindliche Bauer vernichtet und der eigene rückt diagonal ein Feld hinter den nicht mehr existierenden Bauern.}
	\item{Sobald ein Bauer die gegnerische ,,Grundreihe'' erreicht, wird ein Bauerntausch durchgeführt. Hier muss der Bauer gegen Dame, Turm, Läufer oder einen Springer eingetauscht werden.}
\end{itemize}

\subsubsection{Zugregel Springer:}
\begin{itemize}
	\item{Der Springer darf auf das Feld ziehen, dass zwei Felder horizontal bzw. diagonal und eines diagonal bzw. horizontal (gegengleich) versetzt ist. z.B.: Von  b8 auf c6}
\end{itemize}
\subsubsection{Zugregel Läufer:}
\begin{itemize}
	\item{Läufer dürfen diagonal, so weit wie sie wollen ziehen und schlagen. Jedoch darf er nicht über eine Figur ziehen.}
\end{itemize}

\subsubsection{Zugregel Turm:}
\begin{itemize}
	\item{Ein Turm darf horizontal bzw. vertikal ziehen und schlagen wie weit er will, jedoch nicht über Figuren hinweg.}
\end{itemize}

\subsubsection{Zugregel Dame:}
\begin{itemize}
	\item{Eine Dame darf horizontal, vertikal bzw. diagonal ziehen und schlagen so weit wie sie will, jedoch nicht über Figuren hinweg.}
\end{itemize}

\subsubsection{Zugregel König:}
\begin{itemize}
	\item{Der König kann horizontal, vertikal bzw. diagonal ein Feld ziehen.}
	\item{Spezialzug: ,,Rochade''. Dabei wird der König entweder zwei Felder nach links, bzw. zwei Felder nach rechts bewegt. Der Turm bewegt sich dabei drei Felder nach rechts bzw. zwei Fleder nach links. König und Turm dürfen bis zu diesen Zug noch nicht bewegt worden sein, weiters darf keines der Felder über das sie ziehen, der König oder der Turm bedroht werden.}
\end{itemize}


%===========================================================================================

\subsection{Schachmaschinen:}

Seit dem es die Möglichkeit gibt einen Schachspielenden Mechanismus zu bauen, hat man dies auch getan. Zu Anfang war dies noch der ,,schachspielende Türke'', welcher 1769 von Wolfgang von Kempelen konstruiert wurde. \\ 
Der richtige Durchbruch geschah aber erst durch die Erfindung des Computers. Die Hardware wurde immer Leistungsfähiger, wodurch der Mensch als Gegner immer weiter in Bedrängung geriet. 1997 schlug der von IBM speziell entwickelte Schachcomputer Deep Blue, den damaligen Schachweltmeister Kasparow, wodurch die Künstliche Intelligenz in diesem Bereich offiziell den Menschen überholt hat. \\
Heutzutage wird gegen Schachcomputer vor allem zu Trainingszwecken gespielt. Solche Schachcomputer finden sich mittlerweile auf so ziemlich jeden Gerät, egal ob Smartphone, Tablet oder PC/Laptop. Meist sind diese Programme aber Proprietär und ,,closed source''. Wir entwickeln deshalb ein ,,open Source'' Schachspiel, dass auf mehreren Devices spielbar ist.
 



 
%===========================================================================================
\clearpage\vfill\newpage{}
%===========================================================================================


%TODO
\section{Java Chess}
\subsection{Einführung:}

Bevor mit der Dokumentation des Programmcodes begonnen werden kann, werden zuerst einige Möglichkeiten beschrieben, wie ein Schachprogramm prinzipiell programmiert werden kann.

\subsubsection{Repräsentation der Spielfiguren:}

Hierfür gibt es prinzipiell zwei Möglichkeiten:
\begin{enumerate}
	\item{Die Figuren kennen ihre Position}
	\item{Das Brett kennt die Positionen der Figuren}
\end{enumerate}

Das die Figuren ihre Position kennen, klingt zuerst gar nicht so abwegig. Probleme treten aber auf, sobald Schachmatt überprüft werden soll. Hierfür muss überprüft werden ob irgendeine gegnerische Figur den König schlagen kann, wofür man aber das Objekt der Figur benötigt. Natürlich ist dies Programmiertechnisch kein Problem, dadurch entstehen aber längere Wartezeiten. 


Falls das Brett die Position der Figuren kennt und diese Figuren lediglich über eine Zahlenmatrix dargestellt werden, ist das Spiel nicht nur sehr viel performanter, es ergeben sich auch große Vorteile beim entwickeln der Künstlichen Intelligenz. \\ 
Wir entschieden uns für diese Lösung.


\subsection{Repräsentation der Figuren:}

Die Figuren werden über eine Zahlenmatrix repräsentiert. Dabei bekommt jede Figur eine individuelle Zahl zugeteilt. \\
Eine solche Zahl besteht aus 3-Ziffer, z.B.: 102. Diese ist der 2. weiße Bauer, die 1. Ziffer gibt dabei an, ob es Team Weiß (1) oder Schwarz (2) ist. Die 2. Ziffer gibt den Figurentyp an, also Bauer, Turm, etc. Die 3.Ziffer gibt an die wievielte Figur es ist. \\
Diese Matrix ist in einem Objekt von der ,,Background-Matrix'' gespeichert. \\
Zu Beginn einer jeden Partie wird einmal die Startaufstellung im ,,Constructor'' der ;;Background-Matrix'' initialisiert.

\begin{center}
	\begin{tabular}{| c | c | c | c | c | c | c | c |}
		\hline
		110 & 120 	& 	130 & 140 	& 150 	& 131 	& 121 	& 	111 \\ \hline
		101 & 102 	& 	103 & 	104 & 	105 & 	106 & 	107 & 	108 \\ \hline
		0	&	0	& 	0	&	0	&	0	&	0	&	0	&	0	\\ \hline
		0	&	0	& 	0	&	0	&	0	&	0	&	0	&	0 	\\ \hline
		0	&	0	& 	0	&	0	&	0	&	0	&	0	&	0 	\\ \hline
		0	&	0	& 	0	&	0	&	0	&	0	&	0	&	0 	\\ \hline
		201 &	202 &	203	&	204	&	205	&	206	&	207	&	208	\\ \hline
		210 & 	220	&	230	&	240	&	250	&	231	&	221	&	211 \\ 
		\hline	
	\end{tabular}
\end{center}


%=================================================================================
\clearpage\vfill\newpage{}
%=================================================================================
\section{\sc Fertigungsdokumentation}
	{\yhbu
	Die Fertigungsdokumentation ist als optionaler Dokumentationsteil zu sehen und
	wird mit der Betreuerin bzw. Betreuer besprochen, ob dieser Teil der Dokumentation
	notwendig ist. Gedacht ist die Fertigungsdokumentation speziell für aufwändige
	Schaltungen, bei denen es notwendig erscheint, einen Verkabelungsplan, spezielle
	Anleitungen für das Einlöten der Bauteile usw. auszuarbeiten.
	}
\section{\sc Benutzerdokumentation}
	{\yhbu
	Hinweis: Die Benutzerdokumentation beschreibt das System aus der Sicht des
	Benutzers. Ein beliebiger Benutzer sollte in die Lage versetzt werden, das System
	zu verwenden (Bedienungsanleitung, technische Dokumentation).
	}
 \subsection{Installationsanleitung}
	{\yhbu
	Schritt-für-Schritt-Anleitung, wie das System vom Benutzer erstmalig in Betrieb
	genommen werden kann. Weiters eine Anleitung, wie die Software des Systems mit
	Hilfe der Entwicklungswerkzeuge neu erstellt werden kann.
	}
 \subsection{Anwendungsbeispiele}
	{\yhbu
	Beschreibung typischer Aufgaben, die der Benutzer mit dem System durchführen
	kann (Schritt-für-Schritt-Anleitungen).
	}
 \subsection{Referenzhandbuch}
	{\yhbu
	Beschreibung der einzelnen Bedienungselemente (Frontplatten, Dialoge...).
	}
 \subsection{Fehlermeldungen und Hinweise auf Fehlerursachen}
	{\yhbu
	Alle Fehlermeldungen, die das System dem Benutzer ausgeben kann, mit
	Beschreibung der Ursache und Vorschlägen zur Lösung des Problems.
	}









%=================================================================================
\clearpage\vfill\newpage{}
%=================================================================================
\renewcommand{\thesection}{\Roman{section}\;}
%\renewcommand{\labelsection}{(\roman{section})}
\setcounter{section}{0}
%\section{\sc \;}\hfill\\[-24mm]
	%Abbildung 2: Zeitplanung Projekt	8
\section{\sc Abbildungsverzeichnis}\noindent%
	\\ Abbildung 1: Datenübertragungsrate SDRAM
	(Quelle: Einfache IT-Systeme, 2009, S.56) \dotfill vi
	\\ Abbildung 2: Projektzeitplan \dotfill VI
	\\[4mm]{\korr $\to$ \LaTeX\, erstellt mit {\em'\tbs{}listoffigures'}
	Abbildungsverzeichnisse vollautomatisch}


%\section{\sc \;}\hfill\\[-23mm]
	%Tabelle 1: Arbeitsaufstellung	9
\section{\sc Tabellenverzeichnis}\noindent%
	\\ Tabelle 1: Arbeitsaufstellung \dotfill VII
	\\[4mm]{\korr $\to$ \LaTeX\, erstellt mit {\em'\tbs{}listoftables'}
	Abbildungsverzeichnisse vollautomatisch}










%=================================================================================
\clearpage\vfill\newpage{}
%=================================================================================
\renewcommand{\thesection}{\Roman{section}\;}
\setcounter{section}{2}
%\vspace*{-8mm}
\section{\sc Literaturverzeichnis}
	{\yhbu
	Beispiele:
	\\[0mm]{\fontsize{10pt}{10pt}\selectfont
	(Übernommen aus dem Leitfaden des BMBF Reife- und Diplomprüfungen März 2014)
	\\[0mm]
	\begin{description*}
	\item[1. Werke eines Autors] Nachname, Vorname: Titel. Untertitel. -
		Verlagsort: Verlag, Jahr. Nachname,
		Vorname: Titel. Untertitel. Auflage - Verlagsort: Verlag, Jahr.
		\\[1mm]Beispiele:
		\\Sandgruber, Roman: Bittersüße Genüsse. Kulturgeschichte der Genußmittel. – Wien:
		Böhlau, 1986. Messmer, Hans-Peter: PC-Hardwarebuch. Aufbau, Funktionsweise,
		Programmierung. Ein Handbuch nicht nur für Profis. 2. Aufl. - Bonn: Addison-Wesley,
		1993.
		\vspace*{2mm}
	\item[2. Werke mehrerer Autoren] Nachname, Vorname; Nachname, Vorname; Nachname, Vorname: Titel.
		Untertitel. Auflage - Verlagsort: Verlag, Jahr.
		\\[1mm]Beispiel:
		\\Bauer, Leonhard; Matis, Herbert: Geburt der Neuzeit. Vom Feudalsystem zur
		Marktgesellschaft. - Mün- chen: Deutscher Taschenbuch Verlag, 1988.
		\vspace*{2mm}
	\item[3. Sammelwerke, Anthologien, CD-ROM mit Herausgeber] Nachname, Vorname (Herausgeber):
		Titel. Untertitel. Auflage - Verlagsort: Verlag, Jahr. Nachname, Vorname: Titel.
		Untertitel. In: Nachname, Vorname (Herausgeber): Titel. Untertitel. Auflage -
		Verlagsort: Verlag, Jahr.
		\\[1mm]Beispiele:
		\\Popp, Georg (Hg.): Die Großen der Welt. Von Echnaton bis Gutenberg. 3. Aufl. -
		Würzburg: Arena, 1979. Killik, John R.: Die industrielle Revolution in den Vereinigten
		Staaten. In: Adams, Willi Paul (Hg.): Die Vereinigten Staaten von Amerika. Fischer
		Weltgeschichte Bd. 30. - Frankfurt am Main: Fischer Taschenbuch Verlag, 1977. Killy,
		Walther (Hg.): Literatur Lexikon. Autoren u. Werke deutscher Sprache. – München:
		Bertelsmann, 1999. (Digitale Bibliothek, 2)
		\vspace*{2mm}
	\item[4. Mehrbändige Werke] Nachname, Vorname: Titel. Bd. 3 - Verlagsort: Verlag, Jahr.
		\\[1mm]Beispiel:
		\\Zenk, Andreas: Leitfaden für Novell NetWare. Grundlagen und Installation. Bd. 1 - Bonn:
		Addison Wesley, 1990.
		\vspace*{2mm}
	\item[5. Beiträge in Fachzeitschriften, Zeitungen] Nachname, Vorname des Autors des bearbeiteten
		Artikels: Titel des Artikels. In: Titel der Zeitschrift, Heftnummer, Jahrgang, Seite
		(eventuell: Verlagsort, Verlag).
		\\[1mm]Beispiel:
		\\Beck, Josef: Vorbild Gehirn. Neuronale Netze in der Anwendung. In: Chip, Nr. 7, 1993,
		Seite 26. - Würzburg: Vogel Verlag.
		\vspace*{2mm}
	\item[6. CD-ROM-Lexika]\hfill
		\\[1mm]Beispiel:
		\\Encarta 2000 - Microsoft 1999.
		\vspace*{2mm}
	\item[7. Internet] Nachname, Vorname des Autors: Titel. Online in Internet: URL: www-Adresse, Datum.
		(Autor und Titel wenn vorhanden, Online in Internet: URL: www-Adresse, Datum auf
		jeden Fall)
		\\[1mm]Beispiel:
		\\Ben Salah, Soia: Religiöser Fundamentalismus in Algerien. Online im Internet:
		URL: >>http:/\slash{}www.hausarbeiten.de\slash{}cgi-bin\slash{}superRD.pl<<,
		22.11.2000. Der Weg zur Doppelmonarchie.
		Online in Internet: URL:
		http:/\slash{}www.parlinkom.gv.at\slash{}pd\slash{}doep\slash{}d-k1-2.htm,
		22.11.2000.
		\vspace*{2mm}
	\item[8. Firmenbroschüren, CD-ROM] Werden Inhalte von Firmenunterlagen verwendet,
		dann ist ebenfalls die Quelle anzugeben.
		\\[1mm]Beispiel:
		\\Digitale Turbinenregler. Broschüre der Firma VOITH-HYDRO GmbH, 2012.
		\vspace*{2mm}
	\item[9. Abbildungen, Pläne] Werden Abbildungen aus einer fremden Quelle
		[z.B. Download, Scannen) in die Diplomarbeit eingefügt,
		so ist unmittelbar darunter die Quelle anzugeben.
		\\[1mm]Beispiel:
		\\Abb. 1: Digitaler Turbinenregler [ANDRITZ HYDRO]
		\vspace*{2mm}
	\item[10. Persönliche Mitteilungen]\hfill
		\\[1mm]Beispiel:
		\\Persönliche Mitteilung durch: König, Manfred:
		Kössler GmbH Turbinenbau am 8. März 2013.
	\end{description*}
	}}%yhbu













%=================================================================================
\clearpage\vfill\newpage{}
%=================================================================================
\appendix
\part*{\sc Anhang}
\renewcommand{\thesection}{\arabic{section}\;}
\setcounter{section}{5}
\section{\sc Pflichtenheft}
	{\yhbu
	Zur Umsetzung des Projektzieles werden messbare Kriterien formuliert.
	}
 \subsection{Funktionale Anforderungen}
	{\yhbu
	Die Anforderungen müssen detailliert beschrieben werden,
	sie müssen mess- und überprüfbar sein
	(nicht: \dq{}Antwort so schnell wie möglich\dq{}, sondern: \dq{}Antwort in 0.5s\dq{}).
	\\[1mm]
	Sie definieren das Systemverhalten
	(z. B. Erfassen und Interpretieren von Sensorwerten;
	Suchen von Datenbankeinträgen\ldots).
	Die Interaktionen (Mensch-Maschine, Maschine-Maschine)
	sollen auch graphisch dargestellt werden.
	}

 \subsection{Schnittstellen}
	{\yhbu
	Technische Eigenschaften des Systems nach {\em außen}
	(Hardware- und\slash{}oder Softwareschnittstellen).
	Hier wird nur der Verweis auf verwendete Schnittstellen gegeben,
	aber nicht die Definition oder Erklärung
	(außer bei selbst definierten Schnittstellen).
	}

 \subsection{Abnahmekriterien}
	{\yhbu
	Objektive Kriterien,
	mit denen die Vollständigkeit und Korrektheit
	der fertigen Lösung geprüft werden kann
	(z.B.: Bei einem Signalgenerator wird das Ausgangssignal
	in einem definierten Frequenzbereich mit einem Oszilloskop überprüft).
	\\[1mm]
	Die Prototypen der einzelnen Iterationen
	(im Management ist Iteration eine Vorgehensweise,
	um mit den Ungewissheiten und Überraschungen in komplexen Situationen umzugehen)
	sind zu spezifizieren.
	Die Funktionalität der Prototypen muss durch quantifizierbare Ergebnisse überprüfbar sein.
	Die Spezifikation des Prototypen der nächsten Iteration
	ist bei Präsentation einer Iteration vorzustellen.
	Eine Iteration wird mit der Präsentation des Prototypen abgeschlossen (Meilenstein).
	}

 \subsection{Dokumentationsanforderungen}
	{\yhbu
	Anforderungen, die über das Dokument
	\dq{}Gesamtdokumentation\_Vorlage.doc\dq{} hinausgehen
	(z.B.: Online-Hilfe, Dokumentation in Englisch\ldots).
	}

 \subsection{Qualitätsstandards}
	{\yhbu
	Die im Projekt verwendeten Qualitätsstandards und einzuhaltende Normen
	werden festgelegt,
	z.B. hausinterne oder industrielle Printfertigung, verwendete Normen\ldots
	}

 \subsection{Abwicklungsprozess}{\yhbu
	Für die Projektabwicklung an der
	Abteilung für Elektronik und Technische Informatik der HTL-Anichstrasse
	gelten folgende Phasen:
	\\[4mm]\hspace*{-3mm}
	\begin{tabular}{p{50mm} l}
	Startphase Phase \#1
			&(Sommersemester 4. Klasse):	\\
			&Diskussion mit Betreuern -- Erstellung Pflichtenheft	\\
% XH:{\korr do muass wos fahlen:}&\\
	Phase \#2:	&Kontrolle der Meilensteine (fruehestens Oktober)	\\
	Phase \#3:	&Kontrolle der Meilensteine	\\
	Phase \#4:	&Kontrolle der Meilensteine	\\
	Phase \#5:	&Kontrolle der Meilensteine	\\
	Phase \#6:	&Kontrolle der Meilensteine	\\
	Phase \#7:	&04.April 2017 - Abgabe		\\
			&Präsentation und Demonstration der Ergebnisse
	\end{tabular}\hfill
	\\[4mm]
	Die Termine für die Meilensteine werden mit der Betreuerin\slash{}dem Betreuer fixiert.
	\\[2mm]
	Vertiefende Aufgabenstellung laut Antrag.
	\\[2mm]
	Die Erstellung des Pflichtenheftes kann vor September erfolgen, es können auch
	Vorarbeiten in den Ferien erfolgen, jedoch dürfen keine Meilensteine definiert
	werden, die vor und unmittelbar nach der Startphase liegen. Die 150 – 180 Stunden
	sind für das Abschlussjahr definiert.
	}







%==================================================================================
\clearpage\vfill\newpage
%==================================================================================
	% -???
	%iXH weiss nit, wieso da eine Leeseite zu sein hat.
	\vfill
	{\color{white} NIX}




%==================================================================================
\clearpage\vfill\newpage
%==================================================================================
\section{\sc Zusammenfassung}
	{\yhbu
	Kurzbeschreibung in Deutsch, eine A4-Seite.
	Die Zusammenfassung soll eine Einführung in das Thema
	der vertiefenden Aufgabenstellung geben,
	den praktischen Teil kurz beschreiben und die wichtigsten Ergebnisse
	des einzelnen Teammitgliedes anführen.
	Die Zielgruppe der Zusammenfassung sind auch Nicht-Techniker!
	}

 \subsection{Schlussfolgerung / Projekterfahrung}

 \subsection{Projektterminplanung}
	{\yhbu
	Screenshots der MS Project-Datei.
	Die Ausgabe muss lesbar sein (eventuell auf mehrere Bilder verteilen).
	Insbesondere ist darauf zu achten,
	dass die Zeitachse und die Vorgangsachse auf jedem Bild sichtbar sind!
	Es muss nicht MS-Project verwendet werden!
	\\[2mm]
	Projektbalkenplan (Gantt-Diagramm) mit Meilensteinplan:
	}
	%\fbox{\parbox{0.9\linewidth}{ \rule{0pt}{80mm} }}\hfill
	\\[2mm]
	\hspace*{12mm}
	%\includegraphics[width=0.83\linewidth]{imgs/msproj01.png}
	\rule{37mm}{0pt} Abbildung 2: Projektzeitplan

 \subsection{Projektpersonalplanung und Kostenplanung}
	{\yhbu
	Eine Abschätzung von
	{\em Personal, Material, Fremdleistungen}
	und der damit zusammenhängenden {\em Kosten}
	ist in der Vor- bzw. Startphase zu erstellen.
	\\[1mm]
	Grundsätzlich ist die Verfügbarkeit der Ressourcen zu klären.
	ACHTUNG: Es ist zu beachten,
	dass nicht immer alle notwendigen Materialien im Lager
	und Ressourcen (zBsp. Werkstaetten)
	vorhanden bzw. frei sind.
	Die Beschaffung der Materialien ist im Zeitplan mit zu berücksichtigen.
	}

  \subsubsection{Projektkostenplan}
	{\yhbu
	Kalkulation des Gesamtprojektes:
	Personalaufwand (Kosten laut WIR3-Unterricht),
	Kosten für Hard- und Software,
	externe Kosten (z.B.: Sensoren, Bausteine, Kabelkanäle\ldots).
	}

  \subsubsection{Arbeitsnachweis Diplomarbeit}
	{\yhbu
	Dieser erfolgt durch ständige Aufzeichnungen der Schüler im Projekttagebuch.
	\\[1mm]
	Für jeden Projektmitarbeiter wird eine Tabelle gemäß Muster ausgefüllt.
	In dieser Aufzeichnung werden auch die Unterrichtsprojektanteile,
	die in die Arbeit eingeflossen sind, aufgezeigt.
	}

	\paragraph{\color{teal}Tabelle 1: Arbeitsaufstellung}\hfill
	{\fontsize{9pt}{9pt}\selectfont
	\\\begin{tabular}{|l|l|l|p{80mm}|l|}
	\hline
	\multicolumn{5}{|c|}{\parbox{4em}{\hfill\\[-0mm]\color{dkbu}Name}}	\\
	\hline
	Datum	&Uhrzeit	&\parbox{4em}{\hfill\\[-0mm]Stunden\\nn:nn\vspace*{1mm}}
					&Beschreibung	&Betreuer	\\
	\hline
	01.11.2004	&08:00–11:30
				& &Was wurde gemacht (eine Zeile!) &	\\
	\hline
	& & & &\\
	\hline
	& & & &\\
	\hline
	& & SUMME & &	\\
	\hline
	\end{tabular}
	}







  \subsubsection{Leistungscontrolling}
	{\yhbu
	Liefert Informationen über den Fortschritt der Projektleistungserstellung.
	Tabellarische Übersicht über alle Vorgänge: Welche Vorgänge wurden erfüllt,
	welche nicht und warum.
	}




\clearpage\vfill\newpage
	{\yhbu
	\paragraph{\em Abzugeben sind:}\hfill
	\\[1mm]2 gebundene Dokumentationen mit Deckblatt (Format: A4)
	\\[1mm]2 CDs mit allen Unterlagen (Word, Bilder, Code\ldots)
	\\[1mm]2 PowerPoint Folien im HTL Design
	(1. Folie: Vorstellung des Teams und die einzelnen Schwerpunkte
	der Kandidatinnen und Kandidaten.
	2. Folie: Überblick über das Projekt mit Fotos)
	}
	\\[4mm]
	{\yhbu
	Weiters ist vorzubereiten:
	\\[2mm]Ein PowerPoint Vortrag für die Präsentation und Diskussion
	der Diplomarbeit im HTL Design.
	Die Präsentation behandelt nur die Schwerpunkte
	der einzelnen Kandidatin und des Kandidaten.
	Die Teamleiterin\slash{}der Teamleiter gibt eine Gesamtübersicht des Projektes.
	Die Präsentation dauert maximal 8 Minuten/KandidatIn.
	}








\label{LastPage}
%\addtocontents{toc}{\protect\end{multicols}}
\end{document}

%XH 25Feb17:Anpassung gem.YH-neueVorlage 'YH-RbN1-moodle2-Vorlage_DA_sRDP_19102016.docx'
%	RbP:Logo Dicke scalable
%	RbN:Logo
%bis 21Mar17: Warten auf Modifikation Greif-Mikaelyan-Widmann (nicht erhalten)
%XH 21Mar17: Finalisierung (ohne Greif-Mikaelyan-Widmann)
%XH:RdC-1547-2213	Tests variablem '\{0.12}' in 'fancyheader'-Kopfzeilen: vergeblich
%XH:RdD:0857-1112	Text-Check2
%XH:RdF:0914-1737	Preambel-Header kommentieren+ausmisten, Abgleich m. YH'docx'-Version
